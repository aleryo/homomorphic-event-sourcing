%%%%%%%%%%%%%%%%%%%%%%%%%%%%%%%%%%%%%%%%%%%%%%%%%%
\begin{frame}[fragile]{Overview}

\begin{itemize}
\item Introduction
\item Event Sourcing 101
\item Homomorphic Event Sourcing
\item Implementation
\item Conclusion
\end{itemize}

\end{frame}

%%%%%%%%%%%%%%%%%%%%%%%%%%%%%%%%%%%%%%%%%%%%%%%%%%
\section{Introduction}

\begin{frame}[fragile]{Why?}

\begin{center}
{
\LARGE
Why?
}

\vspace{2em}

or:

\vspace{2em}

{
\Large
How this all began
}
\end{center}
\end{frame}


\begin{frame}[fragile]{Who?}

\begin{center}
{
\LARGE
Who we are?
}

\vspace{2em}
\end{center}
\end{frame}


\begin{frame}[fragile]{What?}

\begin{center}
{
\LARGE
Experiment Ideas to Improve Code \& Design
}

\vspace{2em}
\end{center}
\end{frame}

%%%%%%%%%%%%%%%%%%%%%%%%%%%%%%%%%%%%%%%%%%%%%%%%%%
\section{Event Sourcing 101}

\begin{frame}[fragile]{Ubiquitous Language}

\begin{center}
%\includegraphics[height=.4\textheight]{./pics/picture.png}
\end{center}
\end{frame}

\begin{frame}[fragile]{Command, Events, Errors}

  \begin{itemize}
  \item Commands are the queries to a component
  \item Events \& Errors are the replies
  \item Events are persistently stored and represent the state of the system
  \item We can formalize these notions...
  \end{itemize}
\end{frame}

\begin{frame}[fragile]{Base Event Loop}
\begin{center}
\includegraphics[width=\textwidth]{./images/event-loop.pdf}
\end{center}
\end{frame}

\begin{frame}[fragile]{Language}
  \begin{itemize}
  \item $C$ is the set of all Commands, $E$ the set of all Events, $S$ the set of states
  \item $$\delta : S \times C \times E^* \rightarrow S$$ a transition function between states $S$
  \item The language of a component $L \subseteq (C \times E^*)^*$ is a \textbf{transduction}
  \item We can simplify to $L \subseteq ((C + \epsilon) \times E)^*$
  \end{itemize}
\end{frame}

\begin{frame}[fragile]{Language}
  An \emph{Event Sourced} language is such that
  $$\forall s \in S, \exists t \in E^*, \exists c \in C^*, \delta(s_0,c \times t) = s $$
  e.g. the possible states of the system are determined by possible traces of events:
  $$S \subseteq E^*$$
\end{frame}

%%%%%%%%%%%%%%%%%%%%%%%%%%%%%%%%%%%%%%%%%%%%%%%%%%
\part{Homomorphic Event Sourcing}

\begin{frame}[fragile]{Interacting Components}
  \begin{itemize}
  \item What happens when 2 event-sourced components have to interact?
  \item Events  $E_1$ from first component must be transduced into commands $C_2$, and some events $E_2$ into $C_1$
  \item Modelling interaction as a \emph{transformation} of languages
  \end{itemize}
\end{frame}

\begin{frame}[fragile]{Base Event Loop}
\begin{center}
\includegraphics[height=.8\textheight]{./images/interaction-loop.pdf}
\end{center}
\end{frame}

\begin{frame}[fragile]{Homomorphism}

  A \emph{homomorphism} is a structure-preserving map between two languages. This practically means it is enough to
  define a \emph{map} from letters of one language to letters of the other

\end{frame}

\begin{frame}[fragile]{Homomorphic Event Sourcing}
  \begin{itemize}
  \item Define mappings from UI commands to backend commands: An event
    can map to nothing meaning there is not interaction with
    the backend
  \item Define mappings from backend events to UI events: Defines how
    backend's replies are interpreted in the UI
  \item The goal is to define \emph{identities}, e.g. use the same language of commands and events in the frontend and the backendx
  \end{itemize}
\end{frame}

\begin{frame}[fragile]{Benefits}
  \begin{itemize}
  \item Provides a high-level understanding of how of how 2 components
    interact, e.g. a \emph{protocol}
  \item Testing:
    \begin{itemize}
    \item Mock backend to test UI only
    \item Test complete system with respect to overall model
    \item Generate tests from model against UI or backend using the
      defined language as an \emph{oracle} and \emph{generator}
    \end{itemize}
  \item Safer interactions: Define language and homomorphism in a
    single place then generate the other side
  \item Generalization to $n$ components, e.g. hexagonal architecture
  \end{itemize}
\end{frame}

\begin{frame}[fragile]{Beyond Homomorphism}
  \begin{itemize}
  \item Homomorphism do not account for \emph{current state}, e.g. what happened in the past
  \item Some events might be meaningfully interpreted only after some other events happened
  \item \emph{Homomorphism} $\longrightarrow$ \emph{Transduction}
  \end{itemize}
\end{frame}

\begin{frame}[fragile]{State Machines}
  \begin{itemize}
  \item Modelling the \emph{Code Domain}'s behaviour as a \emph{State Machine}
  \item Use the State Machine as a \emph{Generator} to test \emph{Backend}
  \item Use the State Machine as a \emph{Acceptor} to test \emph{Frontend}
  \end{itemize}
\end{frame}

%%%%%%%%%%%%%%%%%%%%%%%%%%%%%%%%%%%%%%%%%%%%%%%%%%
\input cdct

%%%%%%%%%%%%%%%%%%%%%%%%%%%%%%%%%%%%%%%%%%%%%%%%%%
\section{Examples \& Demo}

\begin{frame}[fragile]{Acquire}
  \begin{center}
    \includegraphics[height=.8\textheight]{./images/acquire-boardgame.jpg}
  \end{center}
\end{frame}

\begin{frame}[fragile]{Architecture - Backend}
  \begin{center}
    \includegraphics[height=.8\textheight]{./images/archi-back.pdf}
  \end{center}
\end{frame}

\begin{frame}[fragile]{Architecture - Frontend}
\end{frame}

%%%%%%%%%%%%%%%%%%%%%%%%%%%%%%%%%%%%%%%%%%%%%%%%%%

\begin{frame}[fragile]{Takeaways}
  \only<1>{\Huge \begin{quote} Plans are worthless, but planning is everything \\ \textsc{\Large Dwight D. Eisenhower}\end{quote}}
  \only<2>{\Huge \begin{quote} Models are worthless, but modelling is everything \\ \textsc{\Large Nicole \& Arnaud}\end{quote}}
\end{frame}

\begin{frame}[fragile]{Event Sourcing $\leftrightarrow$ State Machines}
  \begin{itemize}[<+->]
  \item \emph{Event Sourcing} turns a system into an \emph{interpreter} for a language
  \item This makes it easy to model (part of) it as a \emph{State Machine}
  \end{itemize}
  \only<3>{\[
      \begin{array}{c}
        \text{Command} \times \text{State} \rightarrow \text{Event} \times \text{State} \\
         \Updownarrow \\
        \text{State} \xrightarrow{\text{Command} \times \text{Event}} \text{State} \\
      \end{array}
    \]}
\end{frame}

\begin{frame}[fragile]{Caveats}
  \begin{itemize}
  \item It takes time and energy to devise and refine a model
  \item It's not a \emph{Silver Bullet}\textsuperscript{\tiny TM}
  \end{itemize}
\end{frame}


%%%%%%%%%%%%%%%%%%%%%%%%%%%%%%%%%%%%%%%%%%%%%%%%%%
\begin{frame}{Thank you very much!}

  \url{https://github.com/}

  ~\\[1em]
  \begin{block}{Arnaud Bailly}
        \begin{description}[Twitterxx]
        \item[E-Mail]  \href{mailto:arnaud@aleryo.com}{\texttt{arnaud@aleryo.com}}
        \item[Twitter] \href{http://twitter.com/NicoleRauch}{\texttt{@dr\_c0d3}}
        \item[Web] \href{http://aleryo.com}{\texttt{http://aleryo.com}}
        \end{description}
  \end{block}
  \begin{block}{Nicole Rauch}
    \begin{description}[Twitterxx]
    \item[E-Mail]  \href{mailto:info@nicole-rauch.de}{\texttt{info@nicole-rauch.de}}
    \item[Twitter] \href{http://twitter.com/NicoleRauch}{\texttt{@NicoleRauch}}
    \item[Web] \href{http://www.nicole-rauch.de}{\texttt{http://www.nicole-rauch.de}}
    \end{description}
  \end{block}
\end{frame}

%% https://i2.wp.com/thebiggamehunter.com/wp-content/uploads/2011/02/Acquire-3M-Sid-Sackson-11.jpg
