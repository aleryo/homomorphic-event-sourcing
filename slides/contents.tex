%%%%%%%%%%%%%%%%%%%%%%%%%%%%%%%%%%%%%%%%%%%%%%%%%%
\begin{frame}[fragile]{Overview}

\begin{itemize}
\item Introduction
\item Event Sourcing 101
\item A Formal Model for Event Sourcing
\item Consumer-Driven Contract Testing
\item Beyond CDCT
\item Conclusion
\end{itemize}

\end{frame}

%%%%%%%%%%%%%%%%%%%%%%%%%%%%%%%%%%%%%%%%%%%%%%%%%%
\part{Introduction}

\begin{frame}[fragile]{Why?}

\begin{center}
{
\LARGE
Why?
}

\vspace{2em}

or:

\vspace{2em}

{
\Large
How this all began
}
\end{center}
\end{frame}


\begin{frame}[fragile]{Who?}

\begin{center}
{
\LARGE
Who we are?
}

\vspace{2em}
\end{center}
\end{frame}


\begin{frame}[fragile]{What?}

\begin{center}
{
\LARGE
\textbf{Formal Methods} $\cap$ \textbf{DDD}
\\[2em]
Languages, Type Systems, \\[.2em] Engineering Techniques
}

\vspace{2em}
\end{center}
\end{frame}

%%%%%%%%%%%%%%%%%%%%%%%%%%%%%%%%%%%%%%%%%%%%%%%%%%
\part{Event Sourcing 101}

\begin{frame}[fragile]{Ubiquitous Language}

  \begin{itemize}[<+->]
	\item Is $\ldots$ well $\ldots$ ubiquitous
	\item Carries the business language into the code and beyond
	\item Allows everybody to understand everybody else
	\item Understanding without (even unconscious) translation steps
	\item Should be made explicit in a glossary
  \end{itemize}

\end{frame}

\begin{frame}[fragile]{Commands and Events}

  \begin{itemize}[<+->]
  \item Commands represent interactions from the outside world
  \item They are requests to the application
  \item Events  are the system's replies
  \item Events are persistently stored
  \item The current state of the system is the result of all events that happened so far
  \end{itemize}
\end{frame}

\begin{frame}[fragile]{Base Event Loop}
\begin{center}
\includegraphics[width=\textwidth]{./images/event-loop.pdf}
\end{center}
\end{frame}

\begin{frame}[fragile]{Example: An Event-Sourced Pet Store}
  \begin{itemize}[<+->]
  \item Model (part of a) \emph{Pets} online shop
  \item Owner can \emph{Add} some pet or \emph{Remove} it from the store
  \item ``Obvious'' business rules: One cannot add twice the same pet or remove a non existing pet
  \end{itemize}
\end{frame}

\begin{frame}[fragile]{Inputs: Commands \& Queries}
  \begin{lstlisting}[language=Haskell,basicstyle=\ttfamily,keywordstyle=\color{red}]
data Input =
  -- Commands
    Add    { pet :: Pet }
  | Remove { pet :: Pet }
    -- Queries
  | ListPets
\end{lstlisting}
\end{frame}

\begin{frame}[fragile]{Outputs: Events \& Answers}
\begin{lstlisting}[language=Haskell,basicstyle=\ttfamily,keywordstyle=\color{red}]
data Output =
  -- Events
    PetAdded   { pet :: Pet }
  | PetRemoved { pet :: Pet }
  -- Answers
  | Pets       { pets :: [ Pet ] }
  | Error      { reason :: PetStoreError }

-- some errors
data PetStoreError = PetAlreadyAdded
                   | PetDoesNotExist
\end{lstlisting}
\end{frame}

%%%%%%%%%%%%%%%%%%%%%%%%%%%%%%%%%%%%%%%%%%%%%%%%%%
\part{A Formal Model for Event Sourcing}

\begin{frame}[fragile]{Event Sourcing as a Formal Language}

\begin{itemize}[<+->]
\item Conceptually, the commands and events comprise a so-called \textit{alphabet}
\item A \textit{word} is a valid sequence of letters from this alphabet
\item The set of all possible words is a \textit{language}
\item This means each word of the language corresponds to a \textit{state} of the system
\item So the language enumerates all the reachable states of the system
\end{itemize}

\end{frame}

\begin{frame}[fragile]{Example: Pet Store States}
\begin{center}
\includegraphics[height=.8\textheight]{./images/pet-store-iosm.pdf}
\end{center}
\end{frame}


%%%%%%%%%%%%%%%%%%%%%%%%%%%%%%%%%%%%%%%%%%%%%%%%%%
\part{Consumer-Driven Contract Testing}

%%%%%%%%%%%%%%%%%%%%%%%%%%%%%%%%%%%%%%%%%%%%%%%%%%


\begin{frame}[fragile]{Standard Problem for Webapps?}

\begin{itemize}
\item Testing that frontend and backend play nicely together
\end{itemize}

\end{frame}


\begin{frame}[fragile]{The Interaction Problem}
  \begin{center}
    \includegraphics[height=.8\textheight]{./images/interaction-loop.pdf}
  \end{center}
\end{frame}


%%%%%%%%%%%%%%%%%%%%%%%%%%%%%%%%%%%%%%%%%%%%%%%%%%
\begin{frame}[fragile]{Standard Approach: ``Super-Na\"ive''}

\only<1>{
\begin{itemize}
\item Backend designs API
\item Backend is developed, tests are written
\item Frontend waits until backend is implemented
\item Frontend is developed
\item Tests for frontend are written
\item Interaction is tested via integration tests
\end{itemize}
}

\only<2>{
Problems:

\begin{itemize}
\item Frontend development is blocked
\item Integration tests are slow
\item Full integration testing does not scale
\end{itemize}
}

\end{frame}

%%%%%%%%%%%%%%%%%%%%%%%%%%%%%%%%%%%%%%%%%%%%%%%%%%
\begin{frame}[fragile]{Standard Approach: ``Still Quite Na\"ive''}

\only<1>{
\begin{itemize}
\item Backend designs API
\item Frontend writes mocks for backend API
\item Backend is developed, tests are written
\item Frontend can also be developed immediately
\item Interaction tests for frontend use these mocks
\end{itemize}
}

\only<2>{
Problems:

\begin{itemize}
\item Frontend relies on mocks for backend API
\item Do the mocks reflect the backend's actual behaviour?
\item Usually, backend behaviour changes
\item Frontend does not notice because mocks still look good
\end{itemize}
}

\end{frame}

%%%%%%%%%%%%%%%%%%%%%%%%%%%%%%%%%%%%%%%%%%%%%%%%%%
\begin{frame}[fragile]{Standard ``Industry-Strength'' Approach}

\only<1>{
\begin{itemize}
\item Consumer-Driven Contract Testing
\item Hand-written contracts
\end{itemize}
}

\only<2>{
\includegraphics[width=\textwidth]{images/CDCT1.pdf}
}

\only<3>{
\includegraphics[width=\textwidth]{images/CDCT2.pdf}
}

\only<4>{
\includegraphics[width=\textwidth]{images/CDCT3.pdf}
}

\only<5>{
\includegraphics[width=\textwidth]{images/CDCT4.pdf}
}

\only<6>{
\includegraphics[width=\textwidth]{images/CDCT5.pdf}
}

\only<7>{
\includegraphics[width=\textwidth]{images/CDCT6.pdf}
}

\only<8>{
\includegraphics[width=\textwidth]{images/CDCT7.pdf}
}

\end{frame}

\begin{frame}[fragile]{Example: Pet Store Contracts}

\only<1>{
  \lstinputlisting{contracts/nopets}
}

\only<2>{
  \lstinputlisting{contracts/somepets}
}

\end{frame}

\begin{frame}[fragile]{The Problem with CDCT}

\begin{itemize}[<+->]
\item Contract testing is only as good as its contracts
\item Manual contract-writing can be tedious and even error-prone
\item Provider testing must manually establish the desired state
\item Errors may only be discovered late in the process, when the backend implements some functionality and discovers that it does not match the contract
\item If contracts are too sparse, we miss out
\item If contracts are too verbose (or too many), testing takes too long
\end{itemize}

\end{frame}

%%%%%%%%%%%%%%%%%%%%%%%%%%%%%%%%%%%%%%%%%%%%%%%%%%
\part{Beyond CDCT With Model}

\begin{frame}[fragile]{Back to Formal Language}
  \begin{itemize}[<+->]
  \item Often contracts are fairly simple, just mapping requests to replies
  \item What if some request only makes sense in a certain state?
  \item We need more information: Let's use \emph{State Machines}!
  \end{itemize}
\end{frame}


\begin{frame}[fragile]{Our ``formal model'' approach}

  \begin{itemize}[<+->]
  \item Describe the core domain interactions as a formally verifiable model
  \item Generate mocks for the frontend: Use the State Machine as an \emph{Acceptor}
  \item Generate tests for the backend: Use the State Machine as a \emph{Generator}
  \item Guarantee: All aspects of  the model are covered by tests and mocks
  \end{itemize}

\end{frame}

\begin{frame}[fragile]{Model-Based Interaction Testing}
  \only<1>{
    \begin{center}
      \includegraphics[height=.8\textheight]{./images/modelling-interaction-1.pdf}
    \end{center}
  }
  \only<2>{
    \begin{center}
      \includegraphics[height=.8\textheight]{./images/modelling-interaction-2.pdf}
    \end{center}
  }
  \only<3>{
    \begin{center}
      \includegraphics[height=.8\textheight]{./images/modelling-interaction-3.pdf}
    \end{center}
  }
  \only<4>{
    \begin{center}
      \includegraphics[height=.8\textheight]{./images/modelling-interaction-4.pdf}
    \end{center}
  }
  \only<5>{
    \begin{center}
      \includegraphics[height=.8\textheight]{./images/modelling-interaction-5.pdf}
    \end{center}
  }
\end{frame}



\begin{frame}[fragile]{Example: PetStore Model}
\begin{lstlisting}[language=Haskell,basicstyle=\ttfamily,keywordstyle=\color{red}]
petStore :: Input
         -> PetStore
         -> (Maybe Output, PetStore)

petStore Add{pet}  store@PetStore{storedPets}
  | pet `notElem` storedPets =
      (Just $ PetAdded pet,
       store { storedPets = pet:storedPets } )

  | otherwise                =
      (Just $ Error PetAlreadyAdded, store)
\end{lstlisting}
\end{frame}

\begin{frame}[fragile]{Demo}
  \begin{center}
    \Huge{Testing internal consistency of the model}
  \end{center}
\end{frame}

\begin{frame}[fragile]{Demo}
  \begin{center}
    \Huge{Updating the model and testing front-end}
  \end{center}
\end{frame}

\begin{frame}[fragile]{The Code}
  \begin{center}
  \url{https://github.com/aleryo/homomorphic-event-sourcing}
  \end{center}
\end{frame}

\begin{frame}[fragile]{Bug Trophy}
  \begin{itemize}[<+->]
  \item Incorrect routes definitions in the API leading to invalid queries
  \item (\emph{v2}) Forgetting to move basket's content back to store when user logs out
  \item (\emph{v2}) Bad copy/pasting leading to \texttt{RemovePet} actually adding it
  \end{itemize}
\end{frame}

\begin{frame}[fragile]{Real World Application}
  \begin{itemize}[<+->]
  \item Testing Implementation of a Smart Contracts transaction scheduling platform
  \item Define a Model of the system in terms of \emph{Actions}, observable \emph{State} and potential \emph{Failures} from components
  \item Generate sequence of \emph{Action}
  \item Run \emph{Actions} in parallel against the Model and the Implementation
  \item Check reached states in Implementation is identical to the Model's
  \end{itemize}
\end{frame}


%%%%%%%%%%%%%%%%%%%%%%%%%%%%%%%%%%%%%%%%%%%%%%%%%%
\part{Conclusion}

\begin{frame}[fragile]{Event Sourcing \& Formal Methods}
  \begin{itemize}[<+->]
  \item Building an\emph{Event Sourced} system yields opportunities to leverage more formal approaches to V\&V
  \item Modelling as a \emph{State Machine} over a \emph{Formal Language} seems a promising approach
  \item Provides foundations to develop independent parts of the system and \emph{validate} their interaction
  \item It takes time and energy to devise and refine a model!
  \end{itemize}
\end{frame}

\begin{frame}[fragile]{Takeaways}
  \only<1>{\Huge \begin{quote} Plans are worthless,\\ but planning is everything \\ \textsc{\Large Dwight D. Eisenhower}\end{quote}}
  \only<2>{\Huge \begin{quote} Models are worthless,\\ but modelling is everything \\ \textsc{\Large Nicole \& Arnaud}\end{quote}}
\end{frame}

%%%%%%%%%%%%%%%%%%%%%%%%%%%%%%%%%%%%%%%%%%%%%%%%%%
\begin{frame}{Thank you very much!}

  ~\\[1em]
  \begin{block}{Arnaud Bailly}
        \begin{description}[Twitterxx]
        \item[E-Mail]  \href{mailto:arnaud@aleryo.com}{\texttt{arnaud@aleryo.com}}
        \item[Twitter] \href{http://twitter.com/NicoleRauch}{\texttt{@dr\_c0d3}}
        \item[Web] \href{http://aleryo.com}{\texttt{http://aleryo.com}}
        \item[Web] \href{http://symbiont.io}{\texttt{http://symbiont.io}}
        \end{description}
  \end{block}
  \begin{block}{Nicole Rauch}
    \begin{description}[Twitterxx]
    \item[E-Mail]  \href{mailto:info@nicole-rauch.de}{\texttt{info@nicole-rauch.de}}
    \item[Twitter] \href{http://twitter.com/NicoleRauch}{\texttt{@NicoleRauch}}
    \item[Web] \href{http://www.nicole-rauch.de}{\texttt{http://www.nicole-rauch.de}}
    \end{description}
  \end{block}
\end{frame}

%% https://i2.wp.com/thebiggamehunter.com/wp-content/uploads/2011/02/Acquire-3M-Sid-Sackson-11.jpg
